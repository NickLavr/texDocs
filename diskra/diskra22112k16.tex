\documentclass[12pt]{article}
\usepackage[utf8]{inputenc}
\usepackage[russian]{babel}
\usepackage{amsmath}
\usepackage{amsfonts}
\usepackage{amssymb}

\usepackage[margin=1.5cm]{geometry}
\title{Домашная работа по дискретной~математике~№9.\\Графы}
\date{\today}
\author{Лавренов Николай. Группа 161-2, ПМИ}

\newcommand*{\limn}{\ensuremath{\lim\limits_{n\to \infty}}}
\newcommand*{\limto}[2]{\ensuremath{\lim\limits_{#1\to #2}}}
\newcommand*{\limmin}{\ensuremath{\underset{n\to\infty}{\underline{\lim}}}}
\newcommand*{\limmax}{\ensuremath{\overline{\limn}}}
\newcommand*{\limtoinf}[1]{\ensuremath{\lim\limits_{#1\to\infty}}}


\begin{document}
	\section*{№1}
	Дано: последовательность $a_n$, которая задана $a_0 = 5; a_{n+1}=a_n^2+3$.
	Построим вспомогательную последовательность $b_n$ - последняя цифра числа $a_n$.
	Заметим, что каждый элемент (кроме $n = 0$) $b_n$ однозначно задается 
	через $b_{n-1}$ (это следует из делимости чисел на 10).
	
	Таким образом существует не более 10 различных значений $b_n$. 
	
	Рассмотрим начало последовательности $b_n$: \;
	$5, 8, 7, 2, 7, 2, 7, 2, 7, \dots$. Начиная с $b_2$ в последовательности чередуются числа 7 и 2 (это происходит, так как следующий элемент зависит только от предыдушего). 
	Таким образом $b_{2k} = 7, \forall k \in \mathbb{N}$ и
	$b_{2k+1} = 2, \forall k \in \mathbb{N}$.
	В частости, $b_{2015}=2$ %лол, надо было в задаче на 2016 поменять
	
	Ответ: 2
	
	\section*{№2}
	
	Заметим, что каждой правильной раскраске соответствует \
	какая-нибудь последовательность $S \in \{a=\{0, 1, 2\}^n, \forall i \in \mathbb{N}, i < n : a_i \neq a_{i+1}\}$.
	
	Каждая такая последовательность, в свою очередь, однозначно задается 
	первым числом и последовательностью $\{0, 1\}^{n-1}$, указывающей,
	большее из возможных чисел мы берем, или меньшее. 
	Количество различных таких последовательностей	
	$3 \cdot 2^{n-1}$
	(то есть комбинаций из начального числа и какой-то двоичной последовательности длины $n - 1$)
	
	Ответ: $3 \cdot 2^{n-1}$

	%\section*{№3}

	%Поскольку граф связен, существует путь между любыми двумя вершинами.
	%В частости существует путь между двумя (какие-то 2 из 4) вершинами с 
	%нечетной степенью. % #FIXME если он стал несвязным.
	%Обозначим $W$ - множество ребер, которое содержит этот путь.
	%Очевидно, для $W$ существует путь, обходящий все его вершины.
	%Рассмотри граф без этих ребрер. Мы уменьшили степень двух начальных 
	%вершин на 1, значит теперь их степень четная. 
	%Для остальных вершин, входящих в путь степень уменьшилась на два, а,
	%значит, осталась четной. Значит в графе осталось только две вершины с 
	%нечетной степенью. Поэтому существует путь, проходящий через все ребра
	%ровно один раз.
	
	%Заметим, что получившийся путь задает множество ребер $Q$, 
	%не пересеющее с $W$. Поскольку для $W$ и $Q$ существует путь, обходящий
	%их в графе, мы получили необходимое разбиение. Что и требовалось доказать.
	
	\section*{№3}
	
	Возьмем какие-нибудь две вершины с нечетной степенью. 
	Если между ними есть ребро, то обозначим его как один из путей и уберем его из графа.
	Этим действием мы понизили степень двух нечетных вершин на один, сделав ее ченой.
	В графе осталось только две вершины с нечетной степенью, значит в нем существует путь, проходящий
	через все ребра по одному разу.
	
	Однако, ребра между выбранными вершинами могло и не быть. Тогда рассмотрим граф, в котором оно есть.
	Заметим, что добавлением ребра мы повысили на один степень двух нечетных вершин.
	Значит в таком графе остались только две нечетных вершины, и существует путь, проходящий
	через все ребра. Если убрать из этого пути добавленное в граф ребро, то мы получим 2~(не~более) 
	необходимых пути. 
	
	Что и требовалось доказать

	\section*{№4}
	Обозначим $n$ - размерность куба.
	
	Рассмотрим следующий граф: вершинами будут вершины куба, 
	то есть каждой вершине будет соответсвовать элемент из $\{0, 1\}^n$.
	Проведем между вершинами ребро, если между соответствующими вершинами 
	куба есть ребро, другими словами между $u, v \in \{0, 1\}^n$ ребро
	существует тогда и только тогда, когда 
	$ \exists	p: u_p \neq v_p$ и $\forall i \in \mathbb{N}, i \leqslant n,
	i \neq p : u_i=v_i$.
	
	Заметим, что у каждой вершины степень равна $n$ 
	(мы можем поменять ровно один из элементов).
	
	Поккажем по индукции, что всегда можно найти такой путь в графе, 
	который проходжит по каждой вершине ровно 1 раз, 
	причем этот путь может начинаться с любой вершины.
	
	База: если $n = 1$, то в графе всего две вершины, соедененные ребром.
	Путь, соединяющих их подходит. 
	Причем этот путь может начинаться в любой вершине.
	
	Шаг: предположим, что для $k - 1$ найдет искомый путь. 
	Разделим множество вершин в графе на два непересекающихся множества
	в зависимости от элемента вершины $v_1$ 
	(если $v_1 = 0$, будем считать $v \in W$, иначе ($v_1=1$) $v \in Q$).
	Заметим, что из каждой вершины $v \in W$ существует смежная ей вершина $u \in Q$ (а именно та, в которой поменяли $v_1$). 
	Также, по предположения индукции, в подграфах $W$ и $Q$ существует 
	путь, проходящий через все вершины по одному разу. 
	Пусть нам нужно найти такой путь во всем графе, 
	причем он должен начинаться в вершине $e$ 
	(не ограничивая общности будем пологать $e \in W$).
	Для этого возьмем такой путь в $W$, начинающийся с $e$.
	Он заканчивается в какой-то вершине, из которой есть ребро в 
	какую-то вершину $Q$. Соеденим наш путь в $W$ с путем в $Q$ и получим искомый путь в первоначальном графе.
	
	Что и требовалось доказать.
	
	\section*{№5}
	 
	Докажем данное утверждение по индукции от размера графа.
	
	Заметим, что если в графе 1 или 2 вершины, то нужный путь находится.
	
	Рассмотрим какой-нибудь граф $(V, E)$. Возьмем любую вершину $v \in V$.
	
	Найдем все вершины, достижимые из $v$ за один ход,
	обозначим такие вершины как $W$. Также обозначим $Q$ - все вершины,
	достижимые из какой-нибудь $u \in W$ за один ход, то есть достижимые из 
	$v$ за два хода. Пусть $R = V \backslash (Q \cup W \cup \{v\})$.
	
	Если $R = \emptyset$, то это означает, что любая вершина достижима из $v$
	за неболее, чем два ребра. В противном случае заметим, что 
	$\forall o \in R$ проведены ребра $o \rightarrow v$ и 
	$o \rightarrow p \; \forall p \in W$ (иначе проведены обратные ребра,
	что означает принадлежность $o$ к $W$ или $Q$).
	
	Это означает, что для любой вершины из $R$ любая вершина из 
	$V \backslash R$ достижима за не более, чем два хода 
	(для $v$ и $W$ - за один, значит в $Q$ не более, чем за два (через $W$)).
	Для подграфа, индуцированного $(V, E)$ на вершинах $R$ существует вершина,
	из которой достижимы за не более, чем два хода все остальные из $R$ по 
	предположению индукции. Поскольку все остальные вершины для нее 
	также достижимы за нужное число ходов, эта вершина подходит. 
	
	Что и требовалось доказать.	 
	
	\section*{№6}

	Пусть в графе $n$ вершин.	
	
	Докажем по индукции, что можно построить простой ориентированный путь из $k \leqslant n$ вершин.
	
	Заметим, что при $k = 1$ любая вершина образует необходимый путь.
	
	Пусть в графе есть какой-то путь длины $m$. Добавим в этот путь какую-нибудь вершину $v$
	Для этого рассмтрим вершины $u_1, u_2, \dots, u_m$ (в порядке пути). 
	Возможно, что существует ребро $v \rightarrow u_1$. 
	Тогда есть путь длины $m + 1$: $v, u_1, u_2, \dots, u_m$.
	
	В противном случае существует ребро $u_1 \rightarrow v$. 
	Найдем минимальное $1 < i \leqslant m$, что есть ребро $v \rightarrow u_i$. 
	Если такого нет, то, в частности, есть ребро $u_m \rightarrow v$, 
	а значит существует путь длины $m + 1$, который выглядит как
	$u_1, u_2, \dots, u_m, v$. 
	Если же мы нашли нужный $i$, то существует путь длины $m + 1$ вида
	$u_1, \dots, u_{i-1}, v, u_i, \dots, u_m$.
	
	Таким образом в любом случае можно увеличить размер пути. Значит можно сделать такой путь длины $n$.
	Что и требовалось доказать.
	
	\section*{№7}
	
	Докажем следующую лемму: Граф является двураскрашиваемым (двудольным) тогда и только тогда, когда не содержит цикл нечетной длины.
	
	Если в графе есть цикл нечетной длины $v_1, v_2, \dots, v_k$, 
	то при попытке раскрасит в два цвета получим, что смежные $v_1$ и $v_k$ имеют один цвет, противоречие.
	Если нечетных циклов нет, то граф состоит из каких-то четных циклов с присоединенными к ним вершинами.
	Понятно, что можно раскрасить в два цвета как цикли четной длины, так и остальные вершины.
	
	Предположим, что нет циклов нечетной длины. 
	Тогда для каждой компоненты связанности проделаем следующее:
	Выберем какую-нибудь вершину $v$  из нее и, не ограничивая общности, раскрасим ее в цвет $0$.	
	Заметим, что четность любых двух различных путей из $a$ в $b$ совпадает 
	(предположив обратное легко найти цикл нечетной длины, а их в графе быть не должно).
	
	Разделим все вершины (из компоненты) на те, которые достижимы за нечетное число ребер из $v$ и те,
	которые достижимы за четное число ребер из $v$. 
	Если какая-то вершина соединена с $v$ путями чётной длины, 
	то её соседи соединены путями нечётной длины 
	(один такой	путь — через соседа — заведомо есть, а тогда и все пути имеют нечётную длину)
	
	
	Заметим, что в данном графе не более одного цикла нечетной длины. 
	Действительно, возьмем два цикла. Удалим из второго ребро, которого нет в первом.
	Граф по-прежнему не двураскрашиваемый, значит не минимальный.
	
	Предположим, что в графе больше одной компоненты связности с непустым множеством ребер 
	(то есть хотя бы из двух вершин). Но тогда (по предыдущему заключению)
	есть компонента, не содержащая цикла. Удаление любого ребра такой компоненты
	никак не влияет на двураскрашиваемость.
	
	Наконец, предположим, что в графе нет изолированных вершин. 
	Но это значит, что они лежат в одной компоненте.
	Заметим, что все ребра должны участвовать в цикле нечетной длины, иначе их удаление 
	оставит граф недвураскрашиваемым. Это означает, что все вершины входят в цикл.
	Но вершин 1000, а цикл нечетной длины. Значит хотя бы одна вершина в нем не участвует, 
	а по предыдущим пунктам не может быть не изолированной.
	
	Что и требовалось доказать
	
\end{document}







