\documentclass[12pt]{article}
\usepackage[utf8]{inputenc}
\usepackage[russian]{babel}
\usepackage{amsmath}
\usepackage{amsfonts}
\usepackage{amssymb}

\usepackage[margin=1.5cm]{geometry}
\title{Домашная работа по дискретной~математике~№9.\\Графы}
\date{\today}
\author{Лавренов Николай. Группа 161-2, ПМИ}

\newcommand*{\limn}{\ensuremath{\lim\limits_{n\to \infty}}}
\newcommand*{\limto}[2]{\ensuremath{\lim\limits_{#1\to #2}}}
\newcommand*{\limmin}{\ensuremath{\underset{n\to\infty}{\underline{\lim}}}}
\newcommand*{\limmax}{\ensuremath{\overline{\limn}}}
\newcommand*{\limtoinf}[1]{\ensuremath{\lim\limits_{#1\to\infty}}}


\begin{document}
	\section*{№1}
	Дано: последовательность $a_n$, которая задана $a_0 = 5; a_{n+1}=a_n^2+3$.
	Построим вспомогательную последовательность $b_n$ - последняя цифра числа $a_n$.
	Заметим, что каждый элемент (кроме $n = 0$) $b_n$ однозначно задается 
	через $b_{n-1}$ (это следует из делимости чисел на 10).
	
	Таким образом существует не более 10 различных значений $b_n$. 
	
	Рассмотрим начало последовательности $b_n$: \;
	$5, 8, 7, 2, 7, 2, 7, 2, 7, \dots$. Начиная с $b_2$ в последовательности чередуются числа 7 и 2 (это происходит, так как следующий элемент зависит только от предыдушего). 
	Таким образом $b_{2k} = 7, \forall k \in \mathbb{N}$ и
	$b_{2k+1} = 2, \forall k \in \mathbb{N}$.
	В частости, $b_{2015}=2$ %лол, надо было в задаче на 2016 поменять
	
	Ответ: 2
	
	\section*{№2}
	
	Заметим, что каждой правильной раскраске соответствует \
	какая-нибудь последовательность $S \in \{a=\{0, 1, 2\}^n, \forall i \in \mathbb{N}, i < n : a_i \neq a_{i+1}\}$.
	
	Каждая такая последовательность, в свою очередь, однозначно задается 
	первым числом и последовательностью $\{0, 1\}^{n-1}$, указывающей,
	большее из возможных чисел мы берем, или меньшее. 
	Количество различных таких последовательностей  
	$3 \cdot 2^{n-1}$
	(то есть комбинаций из начального числа и какой-то двоичной последовательности длины $n - 1$)
	
	Ответ: $3 \cdot 2^{n-1}$

	\section*{№3}

	Поскольку граф связен, существует путь между любыми двумя вершинами.
	В частости существует путь между двумя (какие-то 2 из 4) вершинами с 
	нечетной степенью. % #FIXME если он стал несвязным.
	Обозначим $W$ - множество ребер, которое содержит этот путь.
	Очевидно, для $W$ существует путь, обходящий все его вершины.
	Рассмотри граф без этих ребрер. Мы уменьшили степень двух начальных 
	вершин на 1, значит теперь их степень четная. 
	Для остальных вершин, входящих в путь степень уменьшилась на два, а,
	значит, осталась четной. Значит в графе осталось только две вершины с 
	нечетной степенью. Поэтому существует путь, проходящий через все ребра
	ровно один раз.
	
	Заметим, что получившийся путь задает множество ребер $Q$, 
	не пересеющее с $W$. Поскольку для $W$ и $Q$ существует путь, обходящий
	их в графе, мы получили необходимое разбиение. Что и требовалось доказать

	\section*{№4}
	Обозначим $n$ - размерность куба.
	
	Рассмотрим следующий граф: вершинами будут вершины куба, 
	то есть каждой вершине будет соответсвовать элемент из $\{0, 1\}^n$.
	Проведем между вершинами ребро, если между соответствующими вершинами 
	куба есть ребро, другими словами между $u, v \in \{0, 1\}^n$ ребро
	существует тогда и только тогда, когда 
	$ \exists  p: u_p \neq v_p$ и $\forall i \in \mathbb{N}, i \leqslant n,
	i \neq p : u_i=v_i$.
	
	Заметим, что у каждой вершины степень равна $n$ 
	(мы можем поменять ровно один из элементов).
	
	Поккажем по индукции, что всегда можно найти такой путь в графе, 
	который проходжит по каждой вершине ровно 1 раз, 
	причем этот путь может начинаться с любой вершины.
	
	База: если $n = 1$, то в графе всего две вершины, соедененные ребром.
	Путь, соединяющих их подходит. 
	Причем этот путь может начинаться в любой вершине.
	
	Шаг: предположим, что для $k - 1$ найдет искомый путь. 
	Разделим множество вершин в графе на два непересекающихся множества
	в зависимости от элемента вершины $v_1$ 
	(если $v_1 = 0$, будем считать $v \in W$, иначе ($v_1=1$) $v \in Q$).
	Заметим, что из каждой вершины $v \in W$ существует смежная ей вершина $u \in Q$ (а именно та, в которой поменяли $v_1$). 
	Также, по предположения индукции, в подграфах $W$ и $Q$ существует 
	путь, проходящий через все вершины по одному разу. 
	Пусть нам нужно найти такой путь во всем графе, 
	причем он должен начинаться в вершине $e$ 
	(не ограничивая общности будем пологать $e \in W$).
	Для этого возьмем такой путь в $W$, начинающийся с $e$.
	Он заканчивается в какой-то вершине, из которой есть ребро в 
	какую-то вершину $Q$. Соеденим наш путь в $W$ с путем в $Q$ и получим искомый путь в первоначальном графе.
	
	Что и требовалось доказать.
	
	\section*{№5}
	 
\end{document}







