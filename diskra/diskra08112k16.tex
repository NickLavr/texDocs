\documentclass[12pt]{article}
\usepackage[utf8]{inputenc}
\usepackage[russian]{babel}
\usepackage{amsmath}
\usepackage{amsfonts}
\usepackage{amssymb}

\usepackage[margin=1.5cm]{geometry}
\title{Домашная работа по дискретной~математике~№8.\\Порядки}
\date{\today}
\author{Лавренов Николай. Группа 161-2, ПМИ}

\newcommand*{\limn}{\ensuremath{\lim\limits_{n\to \infty}}}
\newcommand*{\limto}[2]{\ensuremath{\lim\limits_{#1\to #2}}}
\newcommand*{\limmin}{\ensuremath{\underset{n\to\infty}{\underline{\lim}}}}
\newcommand*{\limmax}{\ensuremath{\overline{\limn}}}
\newcommand*{\limtoinf}[1]{\ensuremath{\lim\limits_{#1\to\infty}}}


\begin{document}
	\maketitle
	\newpage
	\section*{№1}
	Дано: $(A, \leqslant )$, причем 
	\begin{itemize}	
	\item $\forall a \in A: a \leqslant a$;
	\item $\forall a, b, c \in A: $ если $a \leqslant b $ и $b \leqslant c$, то $a \leqslant c$
	\end{itemize}
Задается отношение $(A, \sim)$ как $\forall x, y \in A: x \sim y \Leftrightarrow
	x \leqslant y$ и $y \leqslant x$. Докажем для него рефлексивность, симметричность и транзитивность.
	
	Так как $\forall a \in A: a \leqslant a$, значит $\forall a \in A: a \leqslant a$ и $a \leqslant a$
	$\Leftrightarrow a \sim a$ - рефлексивность.
	
	Также $\forall x, y \in A, x \sim y \Rightarrow x \leqslant y$ и $y \leqslant x \Rightarrow
	y \sim x$ - симметричность.
	
	Заметим, что $\forall x, y, z \in A, x \sim y, y \sim z \Rightarrow
	x \leqslant y, y \leqslant x, y \leqslant z, z \leqslant y \Rightarrow $(из транзитивности $\leqslant$)
	$ x \leqslant z, z \leqslant x \Rightarrow x \sim y$ - транзитивность.
	
	Таким образом  $(A, \sim)$ - отношение эквивалентности.
%	\hbox{\sim}
	\section*{№2}
	В множестве из четырёх элементов может быть не более $\frac{3*4}{2}=6$ различных пар, причем все из них могут быть несравнимы. Ответ: $0, 1, 2, 3, 4, 5$ или $6$.
	\section*{№3}
	Для начала заметим, что подмножеств множества $\{1, 2, 3\}$ всего $2^3=8$. А различных делителей числа $42=2*3*7$ тоже $2^3=8$. Построим биекцию между элементами множеств.
	
	Очевидно, что любой $x$ делитель числа 42, представим как $x=2^a*3^b*7^c$, где $a, b, c \in \{0, 1\}$, причем каждой тройке $a, b, c$ однозначно соответствует какой-то делитель. Построим по числам $a, b, c$ такое множество $Y \subseteq \{1, 2, 3\}$, что $1 \in Y \Leftrightarrow a = 1$, $2 \in Y \Leftrightarrow b = 1$ и $3 \in Y \Leftrightarrow c = 1$. Понятно, что описанный способ построения задает биекцию между делителями числа 42 и подмножествами $\{1, 2, 3\}$.
	
	Осталось показать, что при такой биекции порядки изоморфны. Пусть $x, y: 42|x, 42 |y, y|x$. Это равносильно тому, что $x=2^{a_1}*3^{b_1}*7^{c_1}$ и $y=2^{a_2}*3^{b_2}*7^{c_2}$,	 причем 
	$a_1 \leqslant a_2	, b_1 \leqslant b_2, c_1 \leqslant c_2$, где $a_1, a_2, b_1, b_2, c_1, c_2 \in \{0, 1\}$. Построим множества $Y_1$ и $Y_2$ по описанной выше процедуре. $a_1 \leqslant a_2 \Leftrightarrow Y_1 \cap \{1\} \subseteq Y_2 \cap \{1\}$. Аналогично для $b$ и $c$. Равносильно получаем, что $Y_1 \subseteq Y_2$. Значит данные частичные порядки изоморфны.
	
	Ответ: изоморфны.
	\section*{№4}	
	Предположим, что линейные порядки $\mathbb{Z} + \mathbb{N}$ и $\mathbb{Z} + \mathbb{Z}$ изоморфны. Значит существует биекция $\varphi:\mathbb{Z} + \mathbb{N} \rightarrow \mathbb{Z} + \mathbb{Z}$.
	
	%Заметим, что $\mathbb{N}$ и $\mathbb{Z}$ не изоморфны (хотя бы из-за наличия у $\mathbb{N}$ минимального элемента и его отсутсвие у $\mathbb{Z}$).
	
	Для удобства обозначим различные множества чисел как $A, B, C, D$, чтобы перейти к линейным порядкам 
	$A + B$ и $C + D$($A = \mathbb{Z}, B = \mathbb{N'}, C~=~\mathbb{Z}, D~=~\mathbb{Z'}$).
	
	Возьмем $a = 0 \in A$. Пусть $c = \varphi(a) \in C + D$. Также выберем $b = 0 \in B$. Обозначим $d = \varphi(b) \in C + D$. Из изоморфности и $a < b$ следует $c < d$. Отрезок $[a; b]$ (в $A + B$) содержит бесконечное число элементов, значит (поскольку изоморфизм порядков порождает изоморфизм отрезков) отрезок 
	$[c; d]$ (в $C + D$) также содержит бесконечное число элементов. Это возмножно, только если $c \in C$ и $d \in D$ (ведь $c < d$).
	
	Поскольку $d \in D = \mathbb{Z'}$, $\exists k = d - 1 \in D$. Посмотрим на образ $p = \varphi^{-1}(k)$  в $A + B$, причем $k < d \Rightarrow p < b$. Поскольку $\varphi^{-1}(d) = b$, причем $b$ - минимум в множестве $B$, значит $p \in A$. 
	
	Посмотрим на отрезки $[a; p]$ и $[\varphi(a); \varphi(p)]$. Так как $a, b \in A$, отрезок $[a; p]$ содердит конечное число элементов. Однако $\varphi(a) = c \in C$ и $\varphi(p) = k \in D \Rightarrow$ отрезок $[\varphi(a); \varphi(p)]$ содержит бесконечное число элементов. Получаем противоречие, так как изоморфизм порядков не порождает изоморфизм отрезков. Значит предположение неверно, порядки $\mathbb{Z} + \mathbb{N}$ и $\mathbb{Z} + \mathbb{Z}$ неизоморфны.		
	\section*{№5}
	
	Предположим, что линейные порядки $\mathbb{N} \times \mathbb{Z} = A$ и $\mathbb{Z} \times \mathbb{Z} = B$ изоморфны.
	Значит существует биекция $\varphi: A \rightarrow B$, задающая изоморфизм
	($\forall x, y, \in A: x \leqslant y \Leftrightarrow \varphi(x) \leqslant \varphi(y)$).
	
	Пусть $a = (0; 0) \in A$. Пусть $b = \varphi(a) \in B$ и $b$ задается как $b = (p; q); p, q \in \mathbb{Z}$
	Возьмем элемент $r \in B$, который задается парой $(p - 1, q), r < b$. Тогда отрезок в $B$ $[r; b]$.
	Очевидно, что он содержит бесконечное число элементов. 
	Рассмотрим изоморфный ему отрезок ($c = \varphi^{-1}(r) \in A$, $c < a \Rightarrow c = (0; k), k < 0$)
	$[c; a]$. Он содержит $1 - k$ элементов, в частности конечное число элементов. Но значит эти отрезки не изоморфны. Значит наше предположение неверно и порядки $A$ и $B$ не изоморфны.
	
	Ответ: линейные порядки $\mathbb{N} \times \mathbb{Z}$ и $\mathbb{Z} \times \mathbb{Z}$ не изоморфны.
	
	
	
	\section*{№6}
	В этой задаче под обозначением интервала $(a; b)$ я буду подразумевать только рациональные числа этого интервала, то есть $(a; b) \cap \mathbb{Q}$. Также под порядком на таком интервале я буду иметь в виду порядок, индуцируемый $\mathbb{Q}$ на этот интервал.
	
	Заметим, что частичный порядок на любом интервале рациональных чисел изоморфен частичному порядку на интервале вида $(k\sqrt{2}; (k + 1)\sqrt{2})$, где $k \in N$. Докажем это для интервала $(0; 1)$, для других это будет следовать из транзитивности изоморфности и того, что все интервалы в $\mathbb{Q}$ изоморфны.
	
	Построим биекцию $\varphi: (0; 1) \rightarrow (k\sqrt{2}; (k + 1)\sqrt{2})$, задающую изоморфизм.	
	Возьмем опорное число $p \in (k\sqrt{2}; (k + 1)\sqrt{2})$. В частности $p \in \mathbb{Q}.$	
	Пусть $\varphi(\frac{1}{2})=p$. Возьмем последовательность $a_n=\frac{1}{n + 2}$, $\limtoinf{n}a_n = 0$, $a_i > a_{i + 1}$ и последовательность $b_n$, такую, что $b_n \in (k\sqrt{2}; p)$, $\limtoinf{n}b_n = k\sqrt{2}.$ и $b_i > b_{i + 1}$ (например можно взять последовательность уточняющих десятичных записей числа $k\sqrt{2}$ "сверху").
	
	Для начала возьмем полуинтервалы $[a_1; \frac{1}{2})$ 
	и $[b_1; p)$. Доопределим $\varphi()$, используя их изоморфность. Далее будем поступать итеративно, на $i$-ом шаге: %определим $\varphi(a_i)=b_i$, 
	доопределим $\varphi()$, используя изоморфность полуинтервалов $[a_{i + 1}; a_i)$ и $[b_{i + 1}; b_i)$. Важно, что все полуинтервалы берутся в правильном порядке, т. е. принадлежность разным интервалам однозначно определяет поряок на этих элементах (любой элемент из каждого следующего полуинтервала меньше предыдущих).
	
	Таким образом мы определили изоморфность $(0; \frac{1}{2}]$ и $(k\sqrt{2}; p]$. Аналогичным образом определяем изоморфность $(\frac{1}{2}; 1)$ и $(p; (k + 1)sqrt{2})$. Значит интервал $(0; 1)$ изоморфен $(k\sqrt{2}; (k + 1)\sqrt{2})$.
	
	Из этого следует, что порядок на $\mathbb{Q}$ изоморфен порядку на $(k\sqrt{2}; (k + 1)\sqrt{2}), k \in N$.
	
	Теперь покажем, что $\mathbb{N} \times \mathbb{Q}$ изоморфен $\mathbb{Q}$. 
	$\mathbb{N} \times \mathbb{Q}$ изоморфен бесконечному объединению счетного количества $\mathbb{Q}$. Пронумеруем их натуральными числами. Заметим, что $i$-ый из них изоморфен $(i\sqrt{2}; (i + 1)\sqrt{2})$. Если взять объединение таких интервалов, то мы получим интервал (по-прежнему рациональных чисел) $(0; +\infty)$. Понятно, что он изоморфен $\mathbb{Q}$. По транзитивности получаем изоморфность $\mathbb{N} \times \mathbb{Q}$ и $\mathbb{Q}$. Что и требовалось доказать.
	
	Ответ: изоморфны	
	

	\section*{№7}
	Обозначим $(S, <)$ - лексикографический порядок на множестве бесконечных невозрастающий последовательностей натуральных чисел. Заметим, что порядок линейный (действительно, всегда можно сравнить две последовательности). И у данного порядка существует минимальный элемент - последовательность вида $0, 0, 0, \dots$. 
	
	Докажем, что любая убывающая цепь конечна.
	
	Докажем по индукции, что убывающая цепь из последовательностей, чисела которых, не превосходят $n$ конечна. Заметим, что для того, чтобы все элементы цепи состояли из последовательностей, элементы которых не превосходят $n$ достаточно, чтобы такая цепь начиналась с последовательности, первый элемент которой не превосходит $n$. Действительно, любой другой элемент этой последовательности не превосходит первый элемент и, следовательно, $n$. В то же время первые элементы остальных последовательстей не превоходят $n$, поскольку все эти последовательности меньше. Аналогично элементы всех таких последовательностей не могут быть больше $n$.
	
	База: заметим, что если цепь начинается на $1, 1, 1, \dots, 1, \dots$, то любой меньший элемент выглядит как
	$$ \underbrace{1, 1, \dots, 1}_{p}, 0, 0, \dots$$ Уменьшать $p$ можно конечное число раз, значит такая цепь будет конечна.
	
	Шаг: предположим, что все убывающие цепи из чисел, не превосходящих $k - 1$ конечны. Тогда рассмотрим бесконечные цепи, состоящие из чисел, не превосходящих $k$.
	
	Пусть у нас есть последовательность вида 
	$$ \underbrace{k, k, \dots, k}_{p \in \mathbb{N}}, a_1, a_2, \dots, a_n, \dots; a_n \in [0; k - 1]$$
	$p \in N$ показывает, что число одинаковых $k$ в начале последовательности счетно. Если последовательность выглядела как $k, k, \dots$, то любая меньшая последовательность либо состоит из чисел меньше $k$, либо выглядит как представленная выше.
	
	Заметим, что число $p$ чисел в начале последовательности равных $k$ уменьшится через конечное число элементов цепи. Чтобы доказать это предположим обратное. Но по предположению индукции количество убывающих цепей, начинающихся с $a_1, a_2, \dots, a_n. \dots$ конечно. Значит через конечное количество элементов есть элемент цепи вида $0, 0, \dots, 0$. Любой меньший элемент имеет меньшее $p$. Значит предположение неверно. Получаем, что через счетное количество итераций число $p$ уменьшится. Поскольку уменьшаться натуральное число не может бесконечное число раз, то через счетное число итераций число чисел в начале, равных $k$ станет равным 0, то есть все числа станут меньше $k$. Но по предположению индукции убывающая цепь, начинающаяся с этого элемента конечна.
	
	Таким образом любая убывающая цепь $S$ конечна. Значит порядок на $S$ фундированный.
	
	Ответ: фундированное.
	
	\section*{№8}
	
	Докажем с помощью индукции по $k$, что любая антицепь в множестве $\mathbb{N}^k$  с отношением координатного $(\mathbb{N}^k, \leqslant)$ порядка конечна.
	
	База индукции: если $k = 1$, то отношение линейно, что означает, что антицепь состояит не более, чем из одного элемента. Значит любая антицепь конечна.
	
	Шаг индукции: будем считать, что любое множество антицепей в порядке $(\mathbb{N}^{k - 1}, \leqslant)$ конечно.
	Если множество антицепей пустое, то утверждение доказано.
	В противном случае можно выбрать какой-нибудь опорный элемент вида $(a_1, a_2, \dots, a_k)$.
	Выберем из всех антицепей такие, что на первом месте у них число, меньшее $a_1$ 
	(то есть все цепи вида $b_1, b_2, \dots, b_k; b_1 < a_1; \exists i \in [2; k]: b_i > a_i$, иначе это не элемент аницепи).
	
	Заметим, что для $\forall p < a_1$ подмножество антицепей вида $p, c_1, c_2, \dots, c_{k-1}$ конечно
	по предположению индукции: цепи вида $p, c_1, c_2, \dots, c_{k-1}$ несравнимы тогда и только тогда, когда несравнимы цепи вида $c_1, c_2, \dots, c_{k-1}$, а значит их мощность не может превышать мощность множества антицепей в порядке $(\mathbb{N}^{k-1}, \leqslant)$, а оно конечно. 
	
	Поскольку различных $p < a_1$ конечное количество, то подмножество антицепей с первым элементов, не превосходящим $a_1$ конечно. Повторяя данные рассуждения для всех $a_2, a_3, \dots, a_k$ получаем, что мы рассмотрели каждую антицепь хотя бы раз
	(Нерассмотренные элементы не могут входить в антицепь, так как они сравнимы с опорным элементов, ведь каждый элемент не меньше соответствуещего опорного).
	 Значит множество всех антицепей в $(\mathbb{R}^k, \leqslant)$ - объединение конечного числа конечных множеств антицепей. Что и требовалось доказать.
	 
	 Значит, по индукции, утверждение верно для любого $k \in \mathbb{N}$.
	
	Ответ: нет.	
	
\end{document}







