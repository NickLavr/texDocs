\documentclass[12pt]{article}
\usepackage[utf8]{inputenc}
\usepackage[russian]{babel}
\usepackage{amsmath}
\usepackage{amssymb}
\usepackage{amsfonts}
\title{Домашная работа по дискретной математике. Неделя 7. Мощность множеств-2}
\date{\today}
\author{Лавренов Николай. Группа 161-2, ПМИ}
\usepackage[margin=1.5cm]{geometry}
%short commands
\newcommand{\real}{\mathbb{R}}
%\usepackage[14pt]{extsizes}
%-------

\begin{document}
	\maketitle
	\newpage 
	\section*{№1}
	Ответ: нет.
	Покажем это. Заметим, что множество окружностей с общим центром и всеми возможными положительными вещественными радиусами имеет мощность континум (как минимум из-за равномощности $(0; +\inf)$). Однако у таких окружностей конечное множество центров.
	\section*{№2}
	Понятно, что множество бесконечных двоичных последовательностей, удовлетворяющих условия, не более, чем континуум (хотя бы потому что оно подмножества всех бесконечных двоичных последовательностей).
	
	Покажем, что заданные бесконечные двоичные последовательности хотя бы континуальны. Для этого научимся строить такие последовательности по любой бесконечной двоичной последовательности.
	
	Пусть дана последовательность $ b_0, b_1, \dots, b_n, \dots $. 
	Построим не ее основе последовательность 	
	$b_0, \overline{b_0}, b_1, \overline{b_1}, \dots, b_n, \overline{b_n} \dots $,
	то есть после каждой цифры запишем ее отрицание.
	
	Какую бы подстроку нечетное длины мы не взяли из этой последовательности, она будет состоять из некоторого количества пар из последовательных различных цифр и из "лишнего символа". Пары будут давать одинаковое количество нулей и единиц, а один символ - искомую разницу в 1. Значит наша последовательность удовлетворяет условию.
	Также легко заметить, что из различных бесконечных двоичных последовательностей мы получим раличные последовательности.
	
	Таким образом такие последовательности не менее, чем континуум, в частности не счетные.
	
	Ответ: нет, не счетные
	\section*{№3}
	Обозначим $S$ - множество отрезков. Более строго, $S = \{(a, b) | a, b \in \real, a < b\}$
	Отсюда видно, что $S \subseteq \real \times \real$, ведь 
	$\real \times \real = \{(a, b) | a, b \in \real\}$. Значит $S$ не более, чем континуум.
	
	С другой стороны $\real \subseteq S$. Дейтсвительно, ведь существует инъекция $f: \real \to S$. 
	$f(x)=[x, x + 1]$. Значит, $S$ хотя бы континуум.
	
	Таким образом $S$ имеет мощность континуум.
	\section*{№4}
	Построим взаимно однозначное соответствие между последовательностям из 0, 1, 2 и двоичными последовательностями.
	
	Заметим, что после каждого элемента последовательности 0, 1, 2 можно поставить только 2 цифры дальше, чтобы последовательность удовлетворяла условию, что сумма двух любых последовательных цифр не равна 2. 
	(после 0 нельзя поставить 2, после 1 - 1, после 2 - 0). Значит каждый переход к следующей цифре можно записать, как 0 или 1, в зависимости от того, выбрали мы большее из возможных или меньшее. 
	
	Остается лишь задать начальную цифру. Будем считать, что если двоичная последовательность начинается 
	на $00$, то искомая начинается на $1$, $01~\to~1$, $1~\to~2$ (Мы получили полное префиксное дерево, 
	никакая команда не является префиксом другой, значит преобразование полное и однозначное). 
	
	По алгоритму понятно, как выполнять обратные преобразования. Таким образом мы получили биекцию между бесконечными двоичными последовательностями и данными в задаче последовательностями из 0, 1, 2. Значит они имеют мощность континуум.
	
	Ответ: Континуум
	
	\section*{№5}
	Докажем, что множество невозрастающих бесконечных последовательностей натуральных чисел счетно.
	
	Для начала для числа $k \in \mathbb{N}$ возьмем все такие последовательности, наичающиеся с k. Посмотрим, как устроены такие последовательности:
	$$a_1, a_2, \dots, a_n, b, b, b, \dots$$
	где 	$a_1 = k, a_i \leqslant a_{i+1}, b \in (0; k]$
	
	То есть последовательность можно разбить на две части, $a_1, a_2, \dots, a_n$ и $b, b, \dots$. Последовательность $a_1, a_2, \dots, a_n$ - конечна, значит и множества всех таких последовательностей для всех $n$ не более чем счетно. Также очевидно, что множество последовательностей $b, b, \dots$ также счетно. Таким образом множество всех комбинаций таких последовательностей не более чем счетны.
	
	Таким образом для каждого $k$ множество подходящих последовательностей не более чем счетно. Объединение счетного числа не более чем счетных множеств не более чем счетно, но не континуум.
	
	Ответ: нет
		
	\section*{№6}
	Заметим, что $|\mathbb{Q}|=|\mathbb{N}|$, то есть между ними существует биекция. Поэтому множество биекций $\mathbb{N} \to \mathbb{N}$ равномощно множеству биекций  $\mathbb{N} \to \mathbb{Q}$.
	
	Понятно, что каждая биекция $\mathbb{N} \to \mathbb{N}$ соответствует бесконечной последовательности различных натуральных чисел.
	Зададим однозначное соответствие между ними и континуумом.
	
	1) Бесконечная последовательность различных натуральных чисел равномощна бесконечной последовательности натуральных чисел. Зададим такое соответсвие (от различных к обычным) следующим образом: каждый раз выписывая число, будем зачеркивать его и выписывать его номер среди незачеркнутых. Поскольку числа были различные, то нам не придется зачеркивать два раза одно и то же число. Также очевидно, как по любой последовательности не различных натуральных чисел получить последовательность различных.
	
	2) Понятно, что множество бесконечных последовательностей натуральных чисел равномощно множеству бесконечных последовательностей целых неотрицательных чисел.
	
	3) Построим (однозначно) по каждой бесконечной последовательности неотрицательных чисел бесконечную двоичную последовательность. Для этого будем вместо числа $a_i$ писать 
	$$\underbrace{{1, 1, \dots, 1}}_{a_i чисел}, 0$$
	Таким образом получим последовательность вида
	$$\underbrace{{1, 1, \dots, 1}}_{a_1 чисел}, 0, 
	  \underbrace{{1, 1, \dots, 1}}_{a_2 чисел}, 0, 
	  \dots, 
	  \underbrace{{1, 1, \dots, 1}}_{a_n чисел}, 0, 
	  \dots
	  $$
	Понятно, как получать из любой двоичной последовательности необходимую последовательность чисел.
	
	Таким образом, множество последовательностей различных натуральных чисел равномощно множеству доичных последовательностей чисел, в частности их множество составляет континуум.
	
	Ответ: континуум
	
	\section*{№7}
	
	Да, можно. Рассмотрим семейство графиков функций $f(x)=|x+k|$, где $k \in \real$ на области $x \in [-1; 1]$. Очевидно, что для любого $k$ получается подходящая галочка, причем они не пересекаются. Также мощность множества таких функций, и, соответственно, их графиков $\real$.
	
	\section*{№8}
	Обозначим $C$ - множество всех крестиков.	
	
	Для начала заметим, что каждый крест можно задать с помощью двух точек (на $\real$ координатах). Действительно, пусть это две точки - концы одной из диагоналей квадрата. Тогда построим срединный перпендикуляр к такому отрезку, зададим длину в каждую сторону равную половине отрезка.	
	
	Две точки можно задать c помощью четырех чисел, значит $C \subseteq \real^4$, что равносильно $C \subseteq \real$. То есть множество всех крестов не более, чем континуум.
	
	Размером креста будем называть половину его диагонали, центром креста будем называть точку пересечения его диагоналей.
	
	Рассмотрим множество крестов $S \subseteq C$, размер которых хотя бы $r$.
	Понятно, что центры такого множеста крестов лежат на расстоянии хотя бы $r$ (В противном случаи они бы пересекались).
	Разметим всю плоскость в квадраты со стороной $r/2$. Тогда не существует двух центров крестов, которые лежат в одном квадрате(в противном случае расстояние между ними не более чем 
	$\frac{\sqrt{2}}{2}r < r$ - противоречие).
	
	Все квадраты можно пронумеровать с помощью двух целых координат, выбрав како-нибудь за нулевой, то есть мощность их множества 
	$|\mathbb{Z} \times \mathbb{Z}| = |\mathbb{N}|$. Таким образом квадратов счетное количество, значит подмножество квадратов с центрами крестов не более чем счетное, а значит и самих крестов не более, чем счетное.
	
	Итак, мы получили, что множество крестов размером не менее $r$ - не более чем счетное.
	
	Рассмотрим бесконечную счетную радиусов $1, 1/2, 1/4, \dots, 1/2^n, \dots$. Изначально пересчитаем все кресты, размером больше $1$. Затем при переходе от радиуса $1/2^i$ к $1/2^{i+1}$: множество крестов с размером $(1/2^{i+1} 1/2^i]$ не более, чем счетно. Значит их объединение его с множеством посчитанных до этого крестов также не более, чем счетно.
	
	Таким образом мы пересчитали все кресты. Действительно, для любого креста с размером $P$ найдется шаг итерации, на котором мы его добавим. 
	
	Значит множество крестов не более, чем счетное.
	
	Ответ: невозможно разместить континуум крестов на плоскости.

	\section*{№9}
	
	Известно, что $|A \cup B | =  |\real|$. Докажем, что $|A | = |\real|$ или $|B| = |\real|$.
	
	Заметим, что если $A$ или $B$ более чем континуум, то и их объединение хотя бы больше континуума, что противоречит условию. Значит $A$ и $B$ не более континуума.
	
	Возьмем множество точек единичного квадрата	$S = \{(x, y)|x, y \in [0, 1]\}$. Понятно, что таких точек континуум (так как $|S| = | [0; 1] \times [0; 1] | = |\real^2| = |\real|$), а значит 
	$|A \cup B | = | S|$. Тогда зададим биекцию $f(x, y):S \to A \cup B$ ($x, y \in [0; 1]$).
	
	Возмножно, что $\exists t=t_0 \in [0; 1] : \forall p \in [0, 1]: f(t, p) \in A$. Это значит, 
	что существует инъекция $[0; 1] \to A$, которая задется $g(x) = f(t_0, x), x \in [0; 1]$. Поскольку $|A|$ не более, чем континуум, то $|A| = |\real|$
	
	Однако, если это не так, то значит, что 
	$\forall t \in [0; 1]: \exists p \in [0; 1]: f(t, p) \in B$. Зададим инъекцию $P(x): [0; 1] \to B$ как
	$$P(x) = f(x, p), p \in [0; 1], f(x, p) \in B$$
	Поскольку $|B|$ не более, чем континуум, то $|B | = | \real|$.
	
	Таким образом выполняется, что $|A | = | \real|$ или $|B | = | \real|$. Что и требовалось доказать.
	
\end{document}
































