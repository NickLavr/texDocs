\documentclass[12pt]{article}
\usepackage[utf8]{inputenc}
\usepackage[russian]{babel}
\usepackage{amsmath}
\usepackage{amsfonts}
\usepackage{amssymb}

\usepackage[margin=1.5cm]{geometry}
\title{Домашная работа по дискретной~математике~№9.\\Графы}
\date{\today}
\author{Лавренов Николай. Группа 161-2, ПМИ}

\newcommand*{\limn}{\ensuremath{\lim\limits_{n\to \infty}}}
\newcommand*{\limto}[2]{\ensuremath{\lim\limits_{#1\to #2}}}
\newcommand*{\limmin}{\ensuremath{\underset{n\to\infty}{\underline{\lim}}}}
\newcommand*{\limmax}{\ensuremath{\overline{\limn}}}
\newcommand*{\limtoinf}[1]{\ensuremath{\lim\limits_{#1\to\infty}}}



\begin{document}


	\section*{№20}
	a) Заметим, что $f(x)=|x|$ в $x_0=0$ не имеет производной, в то время как $g(x)=0$ и 
	их произведение $h(x)=f(x)g(x)=0$ имеют производные в $x_0=0$
	b) Пусть $f(x)=|x|$ и $g(x)=\frac{1}{|x|}$. Тогда $h(x)=1$ будет иметь производную

	\section*{№22}
	Известно, что $P_n=1 + 2x + 3x^2 + \dots + nx^{n-1}$. 
	Пусть $A_n=x+x^2+x^3+\cdots+x^n$. Заметим, что $P_n=(A_n)'$.
	В свою очередь, 
	$$A_n=x+x^2+x^3+\cdots+x^n=x(1 + x+x^2+x^3+\cdots+x^{n-1})
	=x\frac{(x-1)(1+x+x^2+x^3+\cdots+x^{n-1})}{x-1}=
	\frac{x(x^n-1)}{x-1}$$
	Используя это, получаем:
	$$P_n=(A_n)'=\left(\frac{x(x^n-1)}{x-1}\right)'=
	\frac{(x^{n+1}-x)'(x-1)+x(x^n-1)(x-1)'}{(x-1)^2}=$$ $$
	\frac{((n+1)x^n-1)(x-1)+x(x^n-1)}{(x-1)^2}=
	\frac{2x^{n+1}+(n(x-1)-1)x^n-2x+1}{(x-1)^2}
	$$
	
	Известно, что $Q_n=1^2 + 2^2x + 3^2x^2 + \dots + n^2x^{n-1}$.
	Заметим, что $Q_n=(xP_n)'$.
	Тогда
	$$Q_n=\left(\frac{x(2x^{n+1}+(n(x-1)-1)x^n-2x+1)}{(x-1)^2}\right)'=$$
	$$=\frac{(n(n+2)(x-1)^2-3x+1)x^n+3x-1}{(x-1)^3}$$	
	\section*{№23, a}
	Известно, что $f(x)=x+e^x$ $x > 0$. $f$ монотонно возрастает (т. к. $x$ и $e^x$ монотонно возрастают).
	Тогда $D(f)=(0; +\infty), E(f) = (1; +\infty)$. 
	Значит есть $g(x)=f^{-1}(x)$ - обратная функция, $D(f)=(1; +\infty), E(f) = (0; +\infty)$.
	
	Причем $f'=1+e^x$, а значит $g'=(f')^{-1}=\frac{1}{1+e^x}$
	\section*{№23, b}
	Дана функция $th x=\frac{e^x+e^{-x}}{2}$
	\section*{№24, a}
	Дано: функция $f$  задана параметрически: $y=a(1-\cos t), x = a(t-\sin t)$.
	Тогда 
	$$f'=\frac{(a(1-\cos t))'}{(a(t-\sin t))'}=
	\frac{a\sin t}{a-a\cos t}=
	\frac{\sin t}{1-\cos t}$$
	\section*{№24, b}
	Дано: функция $f$  задана параметрически: 
	$$y=\arccos\frac{1}{\sqrt{1+t^2}}, x = \arcsin\frac{1}{\sqrt{1+t^2}}$$.
	Тогда 
	$$f'=\frac{(\arccos\frac{1}{\sqrt{1+t^2}})'}{(\arcsin\frac{1}{\sqrt{1+t^2}})'}=
	\frac
	{\arccos'\frac{1}{\sqrt{1+t^2}} \cdot \left( \frac{1}{\sqrt{1+t^2}} \right)'}
	{\arcsin'\frac{1}{\sqrt{1+t^2}} \cdot \left( \frac{1}{\sqrt{1+t^2}} \right)'}=	
	\frac
	{\frac{-1}{\sqrt{1 - \frac{1}{1+t^2}}}}
	{\frac{1}{{\sqrt{1 - \frac{1}{1+t^2}}}}}=-1
	$$
\end{document}
